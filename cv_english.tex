%%%%%%%%%%%%%%%%%%%%%%%%%%%%%%%%%%%%%%%%%% "ModernCV" CV and Cover Letter
% LaTeX Template
% Version 1.1 (9/12/12)
%
% This template has been downloaded from:
% http://www.LaTeXTemplates.com
%
% Original author:
% Xavier Danaux (xdanaux@gmail.com)
%
% License:
% CC BY-NC-SA 3.0 (http://creativecommons.org/licenses/by-nc-sa/3.0/)
%
% Important note:
% This template requires the moderncv.cls and .sty files to be in the same
% directory as this .tex file. These files provide the resume style and themes
% used for structuring the document.
%
%%%%%%%%%%%%%%%%%%%%%%%%%%%%%%%%%%%%%%%%%

%----------------------------------------------------------------------------------------
% PACKAGES AND OTHER DOCUMENT CONFIGURATIONS
%----------------------------------------------------------------------------------------

\documentclass[11pt,a4paper,sans]{moderncv} % Font sizes: 10, 11, or 12. paper sizes: a4paper, letterpaper, a5paper, legalpaper, executivepaper or landscape. font families: sans or roman

\moderncvstyle{casual} % CV theme - options include: 'casual' (default), 'classic', 'oldstyle' and 'banking'
\moderncvcolor{green} % CV color - options include: 'blue' (default), 'orange', 'green', 'red', 'purple', 'grey' and 'black'

% \usepackage{lipsum} % Used for inserting dummy 'Lorem ipsum' text into the template

\usepackage[scale=0.8]{geometry} % Reduce document margins
%\setlength{\hintscolumnwidth}{3cm} % Uncomment to change the width of the dates column
%\setlength{\makecvtitlenamewidth}{10cm} % For the 'classic' style, uncomment to adjust the width of the space allocated to your name

%----------------------------------------------------------------------------------------
% NAME AND CONTACT INFORMATION SECTION
%----------------------------------------------------------------------------------------

\firstname{Marcel}
\familyname{Horlings}

% All information in this block is optional, comment out any lines you don't need
\title{Curriculum vitae}
\address{Helper Molenstraat 14a}{9721BV Groningen}{The Netherlands}
\mobile{+316-48 53 20 73}
\email{m.horlings@hey.com}
\photo[70pt][0.4pt]{pictures/profile_graduation_shorter} % The first bracket is the picture height, the second is the thickness of the frame around the picture (0pt for no frame)
\social[linkedin]{marcel-horlings}
\social[twitter]{FatboyPunk}
\social[github]{FatboyPunk}
%----------------------------------------------------------------------------------------

\begin{document}

\makecvtitle % Print the CV title

%----------------------------------------------------------------------------------------
% EDUCATION SECTION
%----------------------------------------------------------------------------------------


\section{Skills}

%----------------------------------------------------------------------------------------
% PROJECTS SKILLS SECTION
%----------------------------------------------------------------------------------------

\cvitem{-}{HTML, CSS, JavaScript, Git, SQL}
\cvitem{-}{Ruby on Rails, Turbolinks, Elixir, Phoenix, EmberJs}

%----------------------------------------------------------------------------------------
% WORK EXPERIENCE SECTION
%----------------------------------------------------------------------------------------

\section{Experience}

\subsection{Defacto}
\cventry{2017--Current}{Lead Software Developer}{Defacto}{Groningen}{The
Netherlands}{
  In these years, we have built the new version of the main application for
  Defacto, called CAPP LMS\@.
  The LMS, created with Elixir, Phoenix, and Turbolinks, is an application
  that tracks the competence and compliance of employees for mostly healthcare and government institutions.
  It supports 250.000 active users in receive ongoing education they need for
  their jobs. To do this, CAPP interacts with multiple API's to obtain users' results
  and to export the institutions' compliance.
  \\ \\ Tasks and achievements:
  \begin{itemize}%
    \item Created multiple detailed CSV reports with SQL\@.
    \item Open-sourced multiple libraries.
    \item Mentored multiple junior \& medior developers.
    \item Gave talks within the company and at meetups.
    \item Set up a PubSub system for side effects in the app.
  \end{itemize}
}

\cventry{2016--2017}{Medior Software Developer}{}{}{}{
  As a medior software developer, I worked on LearningSpaces, which uses Ruby on Rails,
  EmberJS, and PostgreSQL\@. LearningSpaces created a new way of learning
  between co-workers and is used for performance support and onboarding of
  colleagues.
  I have been a part of every feature from concept to shipping and had a
  leading role in further developing the product.
  \\ \\ Tasks and achievements:
  \begin{itemize}%
    \item Had an advising role for management (E.g., what features to build, who to hire).
    \item Helped developers grow in the technical parts of the job.
    \item Made architectural design decisions.
  \end{itemize}
}

\cventry{2014--2016}{Junior Software Developer}{}{}{}{
  I have learned the ins and outs of Ruby on Rails and how optimizing code for readability
  is often better than optimizing for computers to run.
  \\ \\ Tasks and achievements:
  \begin{itemize}%
    \item Implemented core features using both Ruby on Rails and EmberJS\@.
    \item Gave chat-based customer support.
    \item Did workshops with existing customers.
  \end{itemize}
}

\subsection{Funkido}
\cventry{2019--2019}{Bachelor's thesis \& Software Developer}{Miniem}{Groningen}{The
Netherlands}{
  For my Bachelor's thesis, I created a Ruby on Rails backend for the macOS app
  Funkido. With Funkido, users can train their musical skills by drumming along
  to MIDI files that users found online or have on their computers
  online. Because searching for MIDI files that are good enough to play along to is
  hard, I made a search engine based on known sites that contain these files.
  Data is extracted out of each MIDI and combined with reviews of
  users. I have completed the thesis plus the backend application in my spare time.
  \\ \\ Tasks and achievements:
  \begin{itemize}%
    \item Set up the backend.
    \item Built a web scraper to find MIDI files.
    \item Created a GraphQL API to communicate with the macOS app.
    \item Extracted information out of MIDI files (E.g.\ which instruments,
      beats per minute, length of the song).
  \end{itemize}
}


\subsection{Mondonline}
\cventry{2016--2018}{Software Developer \& co-owner}{Mondonline}{Groningen}{The
Netherlands}{
  I built and ran a webshop for dental products. The webshop used Ruby on Rails
  and processed over 3000 orders. It connected with multiple API's to
  receive payments and to create shipping labels.
}



\section{Education}

\cventry{2010--2019}{Bachelor's degree in Software Engineering}{Hanze University of Applied Sciences}{Groningen}{The
Netherlands}{
  \begin{itemize}%
    \item Received the award for the best Bachelor's thesis of the year.
  \end{itemize}
}  % Arguments not required can be left empty

%----------------------------------------------------------------------------------------
% COMPUTER SKILLS SECTION
%----------------------------------------------------------------------------------------



\section{Blog }

\cventry{}{\href{https://devrain.io}{DevRain.io}}{}{}{}{I wrote about some
structuring I've added to the apps that I worked at.}

%----------------------------------------------------------------------------------------
% LANGUAGES SECTION
%----------------------------------------------------------------------------------------

\section{Languages}

\cvitemwithcomment{Dutch}{Native}{}
\cvitemwithcomment{English}{Fluent}{}

%----------------------------------------------------------------------------------------
% INTERESTS SECTION
%----------------------------------------------------------------------------------------

\end{document}
  \\ \\ Tasks and achievements:
