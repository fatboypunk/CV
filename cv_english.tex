%%%%%%%%%%%%%%%%%%%%%%%%%%%%%%%%%%%%%%%%%% "ModernCV" CV and Cover Letter
% LaTeX Template
% Version 1.1 (9/12/12)
%
% This template has been downloaded from:
% http://www.LaTeXTemplates.com
%
% Original author:
% Xavier Danaux (xdanaux@gmail.com)
%
% License:
% CC BY-NC-SA 3.0 (http://creativecommons.org/licenses/by-nc-sa/3.0/)
%
% Important note:
% This template requires the moderncv.cls and .sty files to be in the same
% directory as this .tex file. These files provide the resume style and themes
% used for structuring the document.
%
%%%%%%%%%%%%%%%%%%%%%%%%%%%%%%%%%%%%%%%%%

%----------------------------------------------------------------------------------------
% PACKAGES AND OTHER DOCUMENT CONFIGURATIONS
%----------------------------------------------------------------------------------------

\documentclass[11pt,a4paper,sans]{moderncv} % Font sizes: 10, 11, or 12. paper sizes: a4paper, letterpaper, a5paper, legalpaper, executivepaper or landscape. font families: sans or roman

\moderncvstyle{casual} % CV theme - options include: 'casual' (default), 'classic', 'oldstyle' and 'banking'
\moderncvcolor{blue} % CV color - options include: 'blue' (default), 'orange', 'green', 'red', 'purple', 'grey' and 'black'

% \usepackage{lipsum} % Used for inserting dummy 'Lorem ipsum' text into the template

\usepackage[scale=0.8]{geometry} % Reduce document margins
%\setlength{\hintscolumnwidth}{3cm} % Uncomment to change the width of the dates column
%\setlength{\makecvtitlenamewidth}{10cm} % For the 'classic' style, uncomment to adjust the width of the space allocated to your name

%----------------------------------------------------------------------------------------
% NAME AND CONTACT INFORMATION SECTION
%----------------------------------------------------------------------------------------

\firstname{Marcel}
\familyname{Horlings}

% All information in this block is optional, comment out any lines you don't need
\title{Curriculum vitae}
\address{Helper Molenstraat 14a}{9721BV Groningen}{The Netherlands}
\mobile{+316-48 53 20 73}
\email{m.horlings@hey.com}
\photo[70pt][0.4pt]{pictures/profile_picture} % The first bracket is the picture height, the second is the thickness of the frame around the picture (0pt for no frame)
%----------------------------------------------------------------------------------------

\begin{document}

\makecvtitle % Print the CV title

%----------------------------------------------------------------------------------------
% EDUCATION SECTION
%----------------------------------------------------------------------------------------


\section{Skills}

%----------------------------------------------------------------------------------------
% PROJECTS SKILLS SECTION
%----------------------------------------------------------------------------------------

\cvitem{-}{HTML, CSS, JavaScript, Git, SQL}
\cvitem{-}{Ruby on Rails, Turbolinks, Elixir, Phoenix, EmberJs}

%----------------------------------------------------------------------------------------
% WORK EXPERIENCE SECTION
%----------------------------------------------------------------------------------------

\section{Experience}

% \subsection{Vocational}
\cventry{2017--Current}{Lead Software Developer}{Defacto}{Groningen}{The
Netherlands}{
  In these years we have built the new version of the main application for Defacto called CAPP LMS.
  This was made with Elixir, Phoenix and Turbolinks. CAPP LMS is an application
  to track the competence and compliancy of mostly healthcare and government institutions.
  It supports 250.000 active users to receive ongoing eductation they need for
  their jobs. To do this CAPP interacts with multiple API's to receive results
  of users and to export the compliancy of the institutions.
  \begin{itemize}%
    \item Created multiple detailed CSV reports with SQL.
    \item Open sourced multiple libraries.
    \item Mentored multiple junior \& medior developers.
    \item Gave talks within the company and at meetups.
    \item Set up a PubSub system for side effects in the app.
  \end{itemize}
}

% \subsection{Vocational}
\cventry{2019--2019}{Bachelor's thesis \& Software Developer}{Miniem}{Groningen}{The
Netherlands}{
  For my Bachelor's thesis I created a Ruby on Rails backend for the MacOS app
  Funkido. With Funkido, users can drum along with MIDI files.
  \begin{itemize}%
    \item Built a webscraper to find MIDI-files.
    \item Created a GraphQL API to communicate with the MacOS app.
    \item Extracted information out of MIDI-files (E.g. which instruments,
      beats per minute, length of the song).
  \end{itemize}
}

\cventry{2016--2017}{Medior Software Developer}{Defacto}{Groningen}{The
Netherlands}{
  Worked as a medior software developer with Ruby on Rails and EmberJS. During
  this period I mostly worked on LearningSpaces, which created a new way of
  learning between co-workers.
  \begin{itemize}%
    \item Had an advising role for management (E.g. what to build, who to hire).
    \item Helped developers grow in the technical parts of the job.
  \end{itemize}
}

\cventry{2016--2018}{Software Developer \& co-owner}{Mondonline}{Groningen}{The
Netherlands}{
  In this period I built and ran a webshop for dental products. The webshop was
  built with Ruby on Rails and processed over 3000 orders.
}

\cventry{2014--2016}{Junior Software Developer}{Defacto}{Groningen}{The
Netherlands}{
  Worked as a junior software developer for LearningSpaces that was built on Ruby on Rails and
  EmberJS.
  \begin{itemize}%
    \item Implemented core features using both Ruby on Rails and EmberJS.
    \item Gave chat-based customer support.
    \item Did workshops with existing customers.
  \end{itemize}
}


\section{Education}

\cventry{2010--2019}{Bachelor's degree in Software Engineering}{Hanze University of Applied Sciences}{Groningen}{The
Netherlands}{
  \begin{itemize}%
    \item Received the award for best Bachelor's thesis of the year.
  \end{itemize}
}  % Arguments not required can be left empty

%----------------------------------------------------------------------------------------
% COMPUTER SKILLS SECTION
%----------------------------------------------------------------------------------------



\section{Blog }

\cventry{}{\href{https://devrain.io}{DevRain.io}}{}{}{}{Here I post blogs about development work
I have done.}

%----------------------------------------------------------------------------------------
% LANGUAGES SECTION
%----------------------------------------------------------------------------------------

\section{Languages}

\cvitemwithcomment{Dutch}{Native}{}
\cvitemwithcomment{English}{Fluent}{}

%----------------------------------------------------------------------------------------
% INTERESTS SECTION
%----------------------------------------------------------------------------------------

\end{document}
